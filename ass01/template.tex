\documentclass{ctexart}

\usepackage{graphicx}
\usepackage{amsmath}

\title{Environment}


\author{ \\ 3220102703 Xu Yang 计算机科学与技术}

\begin{document}

\maketitle


\section{计算机环境}
计算机型号:Lenovo Legion R9000P2021H\\
CPU:AMD Ryzen 7 5800H with Radeon Graphics 3.20GHZ\\
内存大小:16GB\\
硬盘大小:512G+1000G\\
显卡型号:RTX3060\\
\section{Linux}
实现方式:虚拟机,基于VMware 17 Pro,内存6.5GB\\
Linux版本:Ubuntu 22.04.1 TLS\\
额外安装的软件:VScode、texlive-full、doxygen、g++、make、libdealii、paraview、gnuplot\\
编辑器版本:VScode1.79.2;gcc编译器版本:11.3.0\\
评估:在未来的学习和工作中,使用Linux环境对于从事计算机相关的科研和机器人课程(例如ROS)非常重要。Linux提供了强大的科学计算和数据分析工具,以及广泛的机器学习和人工智能框架。在科研中,我可以使用Linux来进行实验和模拟,处理大规模数据集,并利用开源工具和库进行数据分析和可视化。在机器人课程中,ROS是一个基于Linux的开源机器人操作系统,使用Linux环境可以方便地开发、控制和测试机器人。总之,Linux环境在未来的学习和工作中将成为我进行科研和机器人相关任务的基础。


\end{document}
